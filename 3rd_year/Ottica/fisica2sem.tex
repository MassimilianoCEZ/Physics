\documentclass[a4paper,11pt]{article}
\usepackage[german,italian]{babel}
\usepackage{fancyhdr}


\textwidth16cm \textheight24cm \topmargin0mm \headheight0mm
\headsep5mm \oddsidemargin0mm \evensidemargin0mm
\parindent0mm




\usepackage{amsmath}
\usepackage{parskip}
\usepackage{dsfont}
\usepackage{fullpage}
\usepackage{amssymb}
\usepackage{tikz,pgfplots}
\usepackage{cancel}
\usepackage{lmodern}

\usepackage[T1]{fontenc}

\usepackage{theorem}
\usepackage{psfrag}
\usepackage{color}
\usepackage{graphicx}
\usepackage{hyperref}

%\newcommand{\Lim}{\displaystyle\lim}
%\newcommand{\Sum}{\sum\limits}



\hypersetup{
    colorlinks=true,
    linkcolor=black,
    urlcolor=blue,
    pdftitle={Fourier Analysis Documentation},
    pdfpagemode=FullScreen,
}




\begin{document}




\title{Fisica}

\author{Massimiliano Ferrulli}
\date{05.03.2022}



\maketitle

\section*{Fisica secondo anno}
Ottica

\pagebreak




\tableofcontents





\pagebreak


\section{Diffrazione}

Quando le onde colpiscono il bordo di una superficie, o un ostacolo o un'apertura di dimensione simile a quella di \(  \lambda \), la direzione di propagazione di tali onde si disperde e le onde sono soggette a interferenza. Ciò non è altro che diffrazione.

\subsection{Diffrazione da singola fenditura non puntiforme}

Consideriamo un'onda con lunghezza d'onda \( \lambda \) difratta da una lunga fenditura sottile di larghezza \textit{a}, proiettata su uno schermo opaco \textit{B}.

\vspace{1mm}

\begin{minipage}{8cm}
    \includegraphics[scale=0.35]{Primominimo.jpg}
    \end{minipage}
    \begin{minipage}{8cm}
    Lo stesso può essere fatto con l'altro minimo.
    \end{minipage}

    \vspace{2mm}

e con l'approssimazione \(  D >> a \) possiamo ottenere la differenza di percorso dei due raggi:

\begin{center}
    \begin{minipage}{8cm}
    \includegraphics[scale=0.35]{Approssimazione.jpg}
    \end{minipage}

    \vspace{2mm}
\end{center}

sappiamo che la differenza di percorso di due raggi è \( n  \frac{\lambda}{2} \) allora l'interferenza sarà completamente distruttiva.
\pagebreak

Per il primo minimo:
\begin{center}
\[
    \frac{a}{2} \sin(\theta) = \frac{\lambda}{2} \to a sin(\theta) = \lambda
\]
\end{center}

cosa succede se riduciamo l'apertura:

$\bullet$ se \(  a > \lambda  \)  e stringiamo la fenditura mentre \( \lambda \) rimane costante, l'ampiezza dell'angolo aumenta essendo che se \( a \downarrow sin(\theta) \uparrow \text{e dunque} \, \theta \uparrow    \)

$\bullet$ se \(  a = \lambda  \) allora \( \theta = \frac{\pi}{2} \) che è l'angolo delle prime frange scure, vuole dire che il massimo centrale si espande su tutto lo schermo in quanto i minimi di primo ordine delimitano questa fascia bianca.


Secondo minimo: 




\begin{minipage}{8cm}
    \includegraphics[scale=0.2]{secondominimononapp.jpg}
    \end{minipage}
    \begin{minipage}{8cm}
    Lo stesso può essere fatto con l'altro minimo.
    \end{minipage}

    \vspace{2mm}

e con l'approssimazione \(  D >> a \) possiamo ottenere la differenza di percorso dei due raggi:


    \begin{minipage}{8cm}
    \includegraphics[scale=0.2]{secondominimo.jpg}
    \end{minipage}
    \begin{minipage}{8cm}
        \[
            \frac{a}{4} \sin(\theta) = \frac{\lambda}{2} \, \to \, a sin(\theta) = 2 \lambda   
        \]
        \end{minipage}
    \vspace{2mm}

generalizzando la posizione dei minimi sarà:

\begin{center}
    \[
    a sin(\theta) = m \lambda \, \, \text{con} \, \, m = 1, 2, 3, ...
    \]
\end{center}



\pagebreak

se noi vediamo un massimo centrale è dovuto all'equazione:


\begin{center}
    \[
    \vert R_2 - R_1 \vert = m \lambda \, \, \text{e nel minimo centrale i due raggi hanno la stessa lunghezza e dunque:}
    \]
\end{center}
\begin{center}
    \[
            \forall \lambda \, \, \text{abbiamo un massimo}
    \]

\end{center}


Per ottenere i massimi possiamo dividere la fenditura in un numero di volte n dispari. immaginiamoci di dividere la fenditura in 3 e che dalla fenditura si generino 99 fenditure, I raggi provenient dall'intervallo 1-33 e 34-66 saranno in controfase mentre il rimanente \( \frac{1}{3}  \) sarà in fase dato che la luce originaria lo era.

Ciò ci porta a imporre la condizione per cui i raggi provenienti da due intervalli vicini si annullino:

\begin{center}
    \[
    \frac{a}{3} sin(\theta) = \frac{\lambda}{2}   
    \]
\end{center}

\begin{center}
    \[
    a sin(\theta) = (m + \frac{1}{2}) \lambda \, , \, m = 1,2,3...  
    \]
\end{center}




\subsection{Intensità delle frange da interferenza di due fenditure puntiformi }


Scriviamo le due equazioni dei raggi:

\[
S_1(x,t) = Sm \,sin(kx-wt) 
\]
\[
S_2(x,t) = Sm \,sin(kx-wt + \varphi) \, \, \varphi = \text{Differenza di percorso}
\]
l'onda risultante sarà:
\[
2 sm \, sin(kx-wt) cos(- \frac{\varphi}{2})    
\]

ricordando che: 


\[
I \propto A^2
\]

\[
I=4 I_0 \cos^2(\frac{\varphi}{2}) \, \, \textit{dove} \, \, Sm^2 = I_o
\]

\[
\text{Max:} \varphi = 2k\pi \, \, \text{Min:} \varphi = (2k -1)\pi    
\]

\pagebreak

da ciò otteniamo che :

\begin{center}
    \[ 
    \frac{\varphi}{2 \pi} = \frac{\Delta  L}{\lambda} \, \to \, \varphi = \frac{2 \pi}{\lambda} d sin(\theta)
    \]
\end{center}

\[
\frac{\cancel{2\pi}}{\lambda} d sin(\theta) = \cancel{2} k \cancel{\pi}
\]
\[
sin(\theta) d = k \lambda    
\]
dimostrazione dell'approssimazione vista precedentemente a livello geometrico

\vspace{4mm}

\[
\text{grafico di} I=4 I_0 \cos^2(\frac{\varphi}{2})
\]
\begin{center}
    
\begin{minipage}{15cm}
    \includegraphics[scale=0.35]{yi.jpg}
    \end{minipage}
\end{center}

\begin{center}
\begin{minipage}{10cm}
    \includegraphics[scale=0.8]{Screenshot_1.png}
    \end{minipage}
\end{center}

\subsection{Intensità delle frange della diffrazione da singola fenditura }

\begin{center}
    

\begin{minipage}{8cm}
    \includegraphics[scale=0.08]{20220305_171402.jpg}
    \end{minipage}
\end{center}




\subsection{Interferenza da doppia fenditura non puntiforme }

\begin{center}  
\begin{minipage}{15cm}
    \includegraphics[scale=0.6]{diffr512a (0-00-00-00).jpg}
    \end{minipage}
\end{center}


\[
    I_\text{interferenza} = 4I_0 \, cos^2(\frac{\varphi}{2})
\]
\[
I_\text{diffrazione} = \frac{I_m \, sin^2(\frac{\varphi}{2})}{(\frac{\varphi}{2})^2}    
\]
se \( \varphi = \frac{2\pi}{\lambda} a \, sin(\theta)\)

\begin{center}
\[
I_\text{diffrazione + interferenza}  = I_m \, cos^2 (\frac{\varphi_\text{interferenza}}{2}) * \frac{sin^2(\frac{\varphi_{diffrazione}}{2})}{(\frac{\varphi_{diffrazione}}{2})^2}
\]
\end{center}

$\bullet$
se ho una sola fenditura d = 0, dove d è la distanza dalle due fenditure e in questo caso:
\[
\varphi = 0 \, \, \text{dato che} \, \, \varphi = d \, sin(\theta) \frac{2\pi}{\lambda}    
\]
$\bullet$
due fenditure puntiformi a = 0



















\end{document}